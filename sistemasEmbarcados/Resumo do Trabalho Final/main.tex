%%%%%%%%%%%%%%%%%%%%%%%%%%%%%%%%%%%%%%%%%
% Wenneker Assignment
% LaTeX Template
% Version 2.0 (12/1/2019)
%
% This template originates from:
% http://www.LaTeXTemplates.com
%
% Authors:
% Vel (vel@LaTeXTemplates.com)
% Frits Wenneker
%
% License:
% CC BY-NC-SA 3.0 (http://creativecommons.org/licenses/by-nc-sa/3.0/)
% 
%%%%%%%%%%%%%%%%%%%%%%%%%%%%%%%%%%%%%%%%%

%----------------------------------------------------------------------------------------
%	PACKAGES AND OTHER DOCUMENT CONFIGURATIONS
%----------------------------------------------------------------------------------------

\documentclass[14pt]{scrartcl} % Font size

\input{structure.tex} % Include the file specifying the document structure and custom commands

%----------------------------------------------------------------------------------------
%	TITLE SECTION
%----------------------------------------------------------------------------------------



\title{	
	\normalfont\normalsize
	\textsc{Programação de Sistemas Embarcados}\\ % Your university, school and/or department name(s)
	\vspace{25pt} % Whitespace
	\rule{\linewidth}{0.5pt}\\ % Thin top horizontal rule
	\vspace{20pt} % Whitespace
	{\huge Visão de Máquina aplicada a contagem de carros}\\ % The assignment title
	\vspace{12pt} % Whitespace
	\rule{\linewidth}{2pt}\\ % Thick bottom horizontal rule
	\vspace{12pt} % Whitespace
}

\author{\LARGE Rodrigo Bandeira} % Your name

\date{\normalsize\today} % Today's date (\today) or a custom date

\begin{document}

\maketitle % Print the title

%----------------------------------------------------------------------------------------
%	FIGURE EXAMPLE
%----------------------------------------------------------------------------------------

\begin{figure}[h] % [h] forces the figure to be output where it is defined in the code (it suppresses floating)
	\centering
	\includegraphics[width=0.5\columnwidth]{iff.png} % Example image
\end{figure}

\newpage

\section{Resumo}

\subsection{O que já fiz}
Estou em um projeto com Flávio Feliciano sobre Visão de Máquina, e já fiz um código para contagem de quantidade de medicamentos em uma cartela de remédios passando em uma esteira, aonde a única coisa que precisava era uma camera nessa esteira. Os resultados obtidos estão presentes no vídeo presente no \href{https://streamable.com/3e0jtt}{Vídeo}.  \newline
O código utilizado envolveu matemática, uma vez que limiarizei a imagem, e a parte limiarizada, foi usada para contar a quantidade de cor na imagem não limiarizada. A imagem foi considerada como uma matriz qualquer. Para isso, foi necessário uma função que descrevesse o comportamento da caixa de medicamentos. A equação achada está abaixo.

\begin{align} 
	\begin{split}
		a=\frac{\sum_{i=1}^{n}x_{i}y_{i}}{\sum_{i=1}^{n}x_{i}^{2}}
	\end{split}					
\end{align}
\href{https://streamable.com/3g9qkd}{Também já fiz um programa que identificava refletância de placas virtuais, a partir de imagens geradas no blender.}
 

\newpage
Todos as ideias tem a intencao de serem integrados a celulares tipo SmartPhone Android.
\subsection{Ideia 1}
Quero fazer um projeto em que uma camera vai contar a quantidade de carros que passa através de uma rua qualquer, e também a velocidade deles, usando apenas uma camera. Pretendo adicionar melhorias de qualidade de vida, como o contato com um aparelho celular do tipo Smartphone Android. Tive essa ideia ao ver o estágio feito no IFF. \newline
As técnicas a serem utilizadas no projeto vão envolver visão de máquina, e também muito aprendizado de máquina. Pretendo fazer um Deep Learning para aprender a identificar um carro.

\subsection{Ideia 2}
Gostaria de fazer um programa que identificasse a refletancia de placas, só que usando uma camera real e em placas reais, ao contrário do que já fiz no virtual.

\subsection{Ideia 3}
Leitor de placas de veículos, a partir de uma camera.

\subsection{Ideia 4}
Contador de remédios em uma esteira.

\subsection{Ideia 5}
Avaliador de desenhos técnicos!

\end{document}
